%%%%%%%%%%%%%%%%%%
%%%   Template para utilização   %%%
%%%%%%%%%%%%%%%%%%

%Para melhor utilização da ferramenta, utilize o pdfLaTeX+MakeIndex+BibTeX%

\documentclass[12pt,openright,oneside,a4paper,english,french,spanish,brazil]{unifil}

\titulo{Objeto de Aprendizado - Simulador MRI e CT %Atualizar conforme progredir%}
\autor{Rodrigo de Souza Castoldi}
\instituicao{Centro Universitário Filadélfia}
\local{Londrina}
\data{2017}
\preambulo{Ciência da Computação}
\orientador{Prof. Me. Ricardo Inácio Álvares e Silva}

\begin{document}

\frenchspacing

%%%%%%%%%%%%%%%%%%%%%%%%%
%%%%%%Elementos pré-textuais%%%%%%%%%
%%%%%%%%%%%%%%%%%%%%%%%%%

\pretextual

%Capa%

\imprimircapa

%Folha de Rosto%

\makeatletter
\renewcommand{\folhaderostocontent}{
\begin{center}
{\ABNTEXchapterfont\bfseries\MakeTextUppercase{\imprimirautor}}
\vspace*{5cm}
\begin{center}
\ABNTEXchapterfont\bfseries\normalsize\MakeTextUppercase{\imprimirtitulo}
\end{center}
\vspace*{1cm}
\abntex@ifnotempty{\imprimirpreambulo}{%
\hspace{.45\textwidth}
\begin{minipage}{.5\textwidth}
\SingleSpacing
 Trabalho de Dissertação apresentado ao \imprimirinstituicao como parte dos requisitos para obtenção de graduação em \imprimirpreambulo.
{\textnormal{\imprimirorientadorRotulo~\imprimirorientador.}}
\end{minipage}%
}%
\vspace*{\fill}
{\bfseries\large\imprimirlocal}
\par
{\bfseries\large\imprimirdata}
\vspace*{1cm}
\end{center}
}
\makeatother

\imprimirfolhaderosto

\clearpage{\pagestyle{empty}\cleardoublepage}

%Folha de Aprovação%

\begin{folhadeaprovacao}
	\begin{center}
		\ABNTEXchapterfont\textbf{\MakeTextUppercase{\imprimirautor}}
		\vspace*{2cm}
		\begin{center}
			\ABNTEXchapterfont\large\textbf{\MakeTextUppercase{\imprimirtitulo}}
		\end{center}
		\vspace*{2cm}
		Trabalho de Conclusão de Curso apresentado à Banca Examinadora do curso de \imprimirpreambulo do \imprimirinstituicao de \imprimirlocal em cumprimento a requisito parcial para obtenção do título de Bacharel em \imprimirpreambulo.
		\par
		\vspace*{.5in}
		\hspace{.6\textwidth}
		\begin{minipage}{.6\textwidth}
			\begin{center}
\MakeTextUppercase{Aprovado pela \textbf{COMISSÃO EXAMINADORA} em \imprimirlocal, \imprimirdata.}
			\end{center}
		\end{minipage}
			\vspace*{\fill}
			\assinatura{{\imprimirorientador} - Orientador}
			\assinatura{Professor 1 - Examinador} %Insira o nome dos outros examinadores
			\assinatura{Professor 2 - Examinador} 
	\end{center}
\end{folhadeaprovacao}

%Dedicatória (opcional)%

\begin{epigrafe}
\vspace*{\fill}
\begin{flushright}
%%
Essa aqui vai pro ayrton senna, mito das corrida.
\end{flushright}
\end{epigrafe} %possivelmente mudar para dedicatoria no begin/end%

%Agradecimentos (opcional)%

\begin{agradecimentos}

%%Insira seus agradecimentos
Agradeço ao NPI pela oportunidade e espaço dado para o desenvolvimento desta pesquisa e projeto.
Agradeço ao Professor Ricardo novamente, pela ajuda e auxílio durante todo o TCC.
\end{agradecimentos}

%Epígrafe (opcional)%

%%Insira sua epígrafe entre as chaves do \textit{}

\begin{epigrafe}
\vspace*{\fill}
\begin{flushright}
\textit{"Just like you i wanted a way out..." \\
(Elliot Alderson)}
\end{flushright}
\end{epigrafe}

% --- resumo em português ---
	%%Altere as informações do resumo%%
\noindent{Castoldi; Rodrigo. \textbf{\imprimirtitulo}. Trabalho de Conclusão de Curso (Graduação) - \imprimirinstituicao. \imprimirlocal, \imprimirdata.}
\par
\begin{resumo}
	%%Escreva seu resumo na língua vernácula aqui%%
Este artigo discute o uso e desenvolvimento de objetos de aprendizado, propondo o desenvolvimento de simuladores para a área de Radiologia, dos quais seram usados para treinamento do uso e configuração de máquinas de MRI e CT por alunos do curso de Radiologia, Biomedicina e Física Médica.Tais simuladores são classificados como Objeto de Aprendizado, sendo fontes de conhecimento de fácil acesso e baixo custo tanto para instituições de ensino quanto para seus alunos.
\\ %Divisão das alternativas de resumo%
Este artigo discute o uso e desenvolvimento de Objetos de Aprendizado e propõem o desenvolvimento de simuladores que sigam os padrões de Objetos de Aprendizado para que sejam utilizados no treinamento de operadores de Resonância Magnética ou Tomografia Computadorizada de maneira ágil e fácil sem a necessidade da obtenção de máquinas físicas por parte das instituições de ensino.Também descreve  os benefícios e impactos que a aplicação de um Objeto de Aprendizado traz para o ambiente onde é aplicado, bem como a crescente deste tipo de ferramenta que é cada vez mais presente no aprendizado.
\vspace{\onelineskip} \\
	%%Adicione as palavras chaves após os dois pontos '':''%%
\noindent
\textbf{Palavras-chaves}: Objeto de Aprendizado, eLearning, Resonância Magnética, Tomografia Computadorizada, Simulador, Radiologia.
\end{resumo}

% --- resumo em inglês ---
	%%Altere as informações do resumo%%
\noindent{Castoldi; Rodrigo. \textbf{\imprimirtitulo}. Trabalho de Conclusão de Curso (Graduação) - \imprimirinstituicao. \imprimirlocal, \imprimirdata.}
\par
	%%Se o seu resumo não for em inglês, altere o ``Abstract'' e ``english'' abaixo.
\begin{resumo}[Abstract]
\begin{otherlanguage*}{english}
	%%Write your abstract in foreign language here%%
\emph{
Extended abstract in English or another foreign language. 30 lines, simple spacing.
}
\vspace{\onelineskip}
\noindent
	%%Add the keywords after the colon '':''%%
\emph{	
\textbf{Keywords}: Learning Object, eLearning, Magnetic Resonance Imaging, Computed Tomography, Simulator, Radiology.
}
\end{otherlanguage*}
\end{resumo}

%\par
%\vspace{11cm}

\tableofcontents*

  \setlength\absleftindent{0cm}
  \setlength\absrightindent{0cm}
  
  %fonte do ambiente%
  \abstracttextfont{\normalfont\normalsize}

  %intentação e espaçamento entre parágrafos%
  \setlength{\absparindent}{0pt}
  \setlength{\absparsep}{18pt}

%\pdfbookmark[0]{\listfigurename}{lof}
%\listoffigures*
%\cleardoublepage

\textual

\renewcommand{\ABNTEXchapterfont}{\fontfamily{cmr}\fontseries{b}\selectfont}
\renewcommand{\ABNTEXchapterfontsize}{\Large}

\renewcommand{\ABNTEXsectionfont}{\uppercase{\fontfamily{cmr}\fontseries{b}\selectfont}}
\renewcommand{\ABNTEXsectionfontsize}{\large}

\chapter{Introdução}

%Funcionalidade da Radiologia em geral%
A Radiologia tem como papel atívo preparar e operar máquinas de diagnóstico por imagem(neste papel, Resonância Magnética e Tomografia Computadorizada), das quais são essênciais na atualidade para detecção de doenças... continuar amanhã

%Resumo de Objeto de aprendizado%
Objeto de Aprendizado é toda e qualquer ferramenta designada para auxílio do ensino e aprendizado ou de acordo com o IEEE, Objetos de Aprendizados são definidos como "Qualquer entidade, digital ou não digital, que podem ser usadas para aprendizado, educação ou treinamento."[1] e apesar do termo existir a mais de 50 anos, tais ferramentas estão em ascensão, na tentativa de aplicar o aprendizado personalizado com a tecnologia adaptável dos Objetos de Aprendizado. %Diferencial da proposta%

\section{Revisão bibíografica}%%Inserir seção (nível 2)

%%Conteúdo

\chapter{Motivação e Idealização}%%Inserir subseção (nível 3)

%%Conteúdo
Este papel é um side-effect gerado pelo projeto inicial da construção de um simulador de máquinas de resonância magnética, pedido realizado pela professora Juliana do curso de radiologia da UniFil. A instituição UniFil está no seu segundo ano do curso tecnólogo de radiologia e ainda não conta com máquinario para treinamento dos alunos.

\subsection{Ih rapaz}%%Inserir subsubseção (nível 4)

%%Conteúdo

\subsubsubsection{Subsubsubseção}%%Inserir subsubsubseção (nível 5)

%%Conteúdo

\chapter{Elementos Textuais}

\section{Lista Numerada}

\begin{enumerate}
 \item Item 1;
 \item Item 2;
 \item Item 3;
 \item Item 4;
 \item Item 5...
\end{enumerate}

\section{Lista com Marcadores}

\begin{itemize}
 \item Item A;
 \item Item B;
 \item Item C;
 \item Item D...
\end{itemize}
\cleardoublepage
\section{Figura}

\begin{figure}[htb]
	\centering
	%\includegraphics[scale=0.20]{heavy.jpeg} Insira o nome da imagem dentro das chaves%
	\caption{``Respiração Forte''} 	
\end{figure}

\begin{table}[htb]
\caption{Legenda para quadros e tabelas em cima}
\begin{tabular}{|c|c|c|}
\hline
a1 & b1 & c1 \\ \hline
a2 & b2 & c2 \\ \hline
a3 & b3 & c3 \\ \hline
a4 & b4 & c4 \\ \hline
\end{tabular}
\end{table}

Para quadros, utilize a mesma estrutura de tabelas, só que altere a sua formatação.
\begin{quadro}[htb]
\caption{\label{quadro_modelo}Legenda do quadro}
\begin{tabular}{ c c c }
a1 & b1 & c1 \\ 
a2 & b2 & c2 \\ 
a3 & b3 & c3 \\ 
a4 & b4 & c4 \\ 
\end{tabular}
\end{quadro}

\section{Paragrafação}
O primeiro parágrafo já é indentado. Para pular uma linha, deve-se colocar duas barras invertidas \textbackslash \textbackslash. \\
\indent Para indentar um novo parágrafo, inicie-o com \textbackslash indent.\\
\noindent Para iniciar um parágrafo sem indentação, é so utilizar \textbackslash noindent

\section{Nota de Rodapé}
Para utilizar a nota de rodapé, é só utilizar a marcação \textbackslash footnote na frente do seu texto. \footnote{Exemplo de nota de rodapé.} Ele numera automaticamente as notas. \footnote{Outra nota de rodapé.}

\chapter{Outros Elementos}

\section{Código-Fonte}
%É possível alterar a linguagem do código utilizando \lstset{language=nomedalinguagem}
\begin{lstlisting}
/* Block
    comment */
public class Exemplo
{

 public static void main(String args[])
 {
    int i;
 
    // Line comment.
    System.out.println("Hello world!");
 
    for (i = 0; i < 1; i++)
    {
        System.out.println("LaTeX is also great for programmers!");
    }
 }
}
\end{lstlisting}

\section{Termos Matemáticos}

\begin{equation} 
\label{eq:equacao} %Título da equacao
5^2 - 5 = 20
\end{equation}

Descrição da equação \ref{eq:equacao}.

\noindent $\forall x \in X, \quad \exists y \leq \epsilon$
\\
$\cos (2\theta) = \cos^2 \theta - \sin^2 \theta$
\\
$\lim_{x \to \infty} \exp(-x) = 0$
\\
$a \bmod b$
\\
$x \equiv a \pmod b$
\\
$f(n) = n^5 + 4n^2 + 2 |_{n=17}$
\\
$\frac{n!}{k!(n-k)!} = \binom{n}{k}$
\\
$\sum_{i=1}^{10} t_i$
\\
$\int\limits_a^b$

\section{Gráficos Químicos}

\chemfig{(-[:0,1.5,,,draw=none]\scriptstyle\color{red}0)
(-[1]1)(-[:45,1.5,,,draw=none]\scriptstyle\color{red}45)
(-[2]2)(-[:90,1.5,,,draw=none]\scriptstyle\color{red}90)
(-[3]3)(-[:135,1.5,,,draw=none]\scriptstyle\color{red}135)
(-[4]4)(-[:180,1.5,,,draw=none]\scriptstyle\color{red}180)
(-[5]5)(-[:225,1.5,,,draw=none]\scriptstyle\color{red}225)
(-[6]6)(-[:270,1.5,,,draw=none]\scriptstyle\color{red}270)
(-[7]7)(-[:315,1.5,,,draw=none]\scriptstyle\color{red}315)
-0} \\

\chemfig{A-B}\\
\chemfig{A=B}\\
\chemfig{A~B}\\
\chemfig{A>B}\\
\chemfig{A<B}\\
\chemfig{A>:B}\\
\chemfig{A<:B}\\
\chemfig{A>|B}\\
\chemfig{A<|B}\\

\chemfig{C(-[:0]H)(-[:90]H)(-[:180]H)(-[:270]H)} \\

\chemfig{-[:30]-[:-30]-[:30]} \\

\chemfig{-[:30]=[:-30]-[:30]} \\

\chemfig{A*6(-B-C-D-E-F-)} \\

\chemfig{A*5(-B-C-D-E-)} \\

\chemfig{*6(=-=-=-)} \\ 

\chemfig{**5(------)} \\

\chemfig{-(-[1]O^{-})=[7]O} \\

\chemfig{-(-[1]O^{\ominus})=[7]O} \\

\chemfig{-\chemabove{N}{\scriptstyle\oplus}(=[1]O)-[7]O^{\ominus}}

\cleardoublepage

\postextual

%%Colocar as referências conforme as normas da ABNT, somente as utilizadas no trabalho e presentes neste manuscrito.

\begin{thebibliography}{99}

\bibitem{ABNTEX2:2014}
{ABNTEX2; ARAUJO, L. C. \textbf{A classe abntex2}: Documentos técnicos e científicos brasileiros compatíveis com as normas ABNT. Sine loco, v. 1.9.2; 2014.}.

\bibitem{Biazin:2008}
{BIAZIN, D. T. \textbf{Normas da ABNT e padronização de trabalhos acadêmicos}. Londrina: Instituto Filadélfia de Londrina; 2008.}

\bibitem{Buneman:2011}
{BUNEMAN, P.; CHENEY, J.; LINDLEY, S. et al. \textbf{DBWiki}: A Structured Wiki for Curated Data and Collaborative Data Management. Athens: SIGMOD’11; 2011.}

\bibitem{Wikibooks:2014}
{WIKIBOOKS. \textbf{LaTeX}: The Free Textbook Project. Disponível em: <http://en.wikibooks.org/wiki/LaTeX>. Acesso em: 09 abr. 2014.}

\end{thebibliography}

\end{document}
